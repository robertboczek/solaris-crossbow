% Commands:
%
% * EtherstubManager._create( name );
% * VnicManager._create( name, temporary, parent );
%
% * one worker instantiation: switch + 2 appliances + communication     (sim)
% * two workers instantiation: (dou)


\documentclass{beamer}

\usepackage[polish]{babel}
\usepackage[utf8]{inputenc}
\usepackage[T1]{fontenc}
\usepackage{hyperref}
\usepackage{graphicx}
\usepackage{multicol}
\usepackage{tabularx}
\usepackage{tikz}

\usetikzlibrary{positioning}

\mode<presentation>{\usetheme{Rochester}}
\setbeamercovered{dynamic}

\title{Component-based system for management of multilevel virtualization of networking resources}
\subtitle{System komponentowy wspomagający wielopoziomową wirtualizację zasobów sieciowych}
\author{Robert Boczek \and Dawid Ciepliński}
\institute{AGH University of Science and Technology \\ ~ \\ Faculty of Electrical Engineering, Automatics, Computer Science and Electronics \\ ~ \\ Department of Computer Science \\ ~ \\ Kraków, Poland}
\date{28.09.2011}

\begin{document}

	\begin{frame}
		\titlepage
	\end{frame}


	\begin{frame}{Agenda}

		\begin{itemize}
			\item<1-> General information
			\item<2-> Motivation
			\item<3-> Thesis aim
			\item<4-> The CM4J system presentation ( Core, Gui )
			\item<5-> Tests results
			\item<6-> Summary
		\end{itemize}

	\end{frame}

	\begin{frame}{General information}

		\begin{itemize}
			\item Project initailly started //tutaj bys napisal co robiles u jarzaba i jak zetknales sie z jimsem
			
			\item Supervisor: \textbf{prof. dr hab. inż. Krzysztof Zieliński}
			\item Technical supervisor: \textbf{mgr Marcin Jarząb}
		\end{itemize}

	\end{frame}

	\begin{frame}{Motivation}

		\begin{itemize}
			\item Interest in distributed systems, computer networking
			\item Lack of applications offering creation of virtualized networks
			\item Desire to learn the crossbow library and Solaris OS
		\end{itemize}

	\end{frame}

	\begin{frame}{Thesis aim}

		\begin{itemize}
			\item "There exists a component-based architecture which enables construction of a system that would facilitate working with fully isolated virtualized network resources grouped in projects"
		\end{itemize}

	\end{frame}

	\begin{frame}{The CM4J system presentation}

		Tu generalnie o stworzonym systemie

	\end{frame}

	\begin{frame}{The CM4J system presentation - GUI}

		\begin{columns}[c]
		\column{1.5in}
			\begin{itemize}
				\item Designing desired network structure with requested virtual appliances,
				\item Discovering and modifying already created projects,
				\item Monitoring
				\item Automatic logging using Secure Shell (SSH)
			\end{itemize}
		\column{1.5in}
			\framebox{\includegraphics[width=1.5in]{img/gui-design.png}}
		\end{columns}

	\end{frame}

	\begin{frame}{Tests results}

		
		\begin{itemize}
			\item Prepared multimedia test case
			\item Streaming server and VOD server
			\item Client differentiation
			\item Different scenarios
		\end{itemize}

	\end{frame}

	\begin{frame}{Tests results}

		\begin{itemize}
			\item Prepared multimedia test case
			\item Streaming server and VOD server
			\item Client differentiation
			\item Different scenarios
		\end{itemize}

	\end{frame}

	\begin{frame}{Tests results}
		
		\begin{figure}
		   \includegraphics[width=0.65\textwidth]{img/diagram.pdf}
		\end{figure}

	\end{frame}

	\begin{frame}{Tests results}
		
		Wstawic wykresy, tabele potwierdzajace dzialanie systemu, opisac ze system zostal stworzony z gui z uprzednio stworzonych snapshotow, generalnie testowany rowniez z gui przez shella

	\end{frame}

	\begin{frame}{Summary}

		
		\begin{itemize}
			\item Prepared complete software system 
			\item Met every production process step: requirements analysis, feasibility analysis, architecture design, implementation, test
			\item Master of science thesis creation
			\item Thesis statement proved by performed tests
			\item Expectations for future system's utilization together with JIMS
		\end{itemize}

	\end{frame}



\end{document}

