\documentclass{beamer}

\usepackage[polish]{babel}
\usepackage[utf8]{inputenc}
\usepackage[T1]{fontenc}
\usepackage{hyperref}
\usepackage{graphicx}
\usepackage{multicol}
\usepackage{tabularx}

\mode<presentation>{\usetheme{Dresden}}
\setbeamercovered{dynamic}

\title{OpenSolaris, Crossbow and JIMS}
\author{Robert Boczek \and Dawid Ciepliński}
\date{26.10.2010}

\begin{document}

\begin{frame}

	\titlepage

\end{frame}


\section{Domain}

	\subsection{}

		\begin{frame}{Crossbow components and their relationship}
		
			\begin{figure}[H]
				\includegraphics[width=\textwidth]{img/domain.png}
			\end{figure}
		
		\end{frame}


\section{Customizable properties}

	\subsection{}

		\begin{frame}{Available link operations}

			\begin{itemize}
				\item Plumbing
				\item Putting interface up/down
				\item Setting IP address
				\item Setting mask address
				\item Getting parent link name				
			\end{itemize}
		\end{frame}


		\begin{frame}{Etherstub and Link (Vnic, Nic) parameters }

			Read-only parameters:
			
			\begin{itemize}

				\item \textbf{BRIDGE} - The name of the bridge to which this link is assigned, if any
				\item \textbf{OVER} - The physical datalink(s) over which the datalink is operating
				\item \textbf{STATE} - The link state of the datalink. The state can be up, down, or unknown
				\item \textbf{MTU} - The maximum transmission unit size for the datalink being displayed
				\item \textbf{CLASS} - The class of the datalink. dladm distinguishes between the following classes:

					\begin{itemize}
						\item \textbf{phys} - A physical datalink. 
						\item \textbf{vnic} - A virtual network interface. 				
					\end{itemize}			
			\end{itemize}

		\end{frame}

		\begin{frame}{Etherstub and Link (Vnic, Nic) properties }

			Editable properties:
			
			\begin{itemize}

				\item \textbf{maxbw} - The full duplex bandwidth specified as an integer with one of the scale suffixes (K, M, or G for Kbps, Mbps, and Gbps). The default is \textbf{no bandwidth limit}
				\item \textbf{learn\_limit} - Limits the number of new or changed MAC sources to be learned over a bridge link. The default is \textbf{1000}
				\item \textbf{cpus} - Names of processors that can perform operations for this link. The default is \textbf{no CPU binding}
				\item \textbf{priority} - Relative priority for the link. Possible values are: \textbf{high}, \textbf{medium}, or \textbf{low}. The default is \textbf{high}
			
			\end{itemize}

		\end{frame}

		\begin{frame}{Etherstub and Link (Vnic, Nic) statistics }

			Read-only statistics:
			
			\begin{itemize}

				\item \textbf{IPACKETS} - Number of packets received on this link
				\item \textbf{RBYTES} - Number of bytes received on this link
				\item \textbf{IERRORS} - Number of input errors
				\item \textbf{OPACKETS} - Number of packets sent on this link
				\item \textbf{OBYTES} - Number of bytes sent on this link
				\item \textbf{OERRORS} - Number of output errors
			
			\end{itemize}

		\end{frame}


		\begin{frame}{Flows}

			Bandwidth control and priority for protocols, services and containers.

			\medskip

			Individual flow restrictions:

			\begin{itemize}
				\item \textbf{local\_port}, \textbf{remote\_port}
				\item \textbf{transport} - \textbf{tcp}|\textbf{udp}|\textbf{sctp}|\textbf{icmp}|\textbf{icmpv6}
				\item \textbf{dsfield}
				\item \textbf{local\_ip[/prefix\_len]}, \textbf{remote\_ip[/prefix\_len]}
			\end{itemize}

			\medskip

			Flow restrictions:

			\begin{itemize}
				\item \textbf{maxbw} - The full duplex bandwidth.
				\item \textbf{priority} - Relative priority for the link.
			\end{itemize}

		\end{frame}


\section{Architecture}

	\subsection{Overview}

		\begin{frame}{Layers and responsibility}

			\begin{figure}[H]
				\includegraphics[width=\textwidth]{img/layers.png}
			\end{figure}
			
		\end{frame}
	

	\subsection{Layers}

		\begin{frame}{lib*adm}

			Crossbow Project libraries.

			\begin{itemize}
				\item \texttt{libdladm} provides API to manipulate VNICs, etherstubs and NICs
				\item \texttt{libflowadm} allows flow management (descriptors - e.g. addresses, protocols, ports, QoS - flow priority and maximum bandwidth)
			\end{itemize}

			Simple operations: \texttt{dladm\_set\_flowprop}, \texttt{dladm\_vnic\_delete}, etc.

		\end{frame}
	

		\begin{frame}{Native wrappers}

			Exploit Crossbow lower-level *adm libraries to provide more complex functionality.
			
			\medskip

			3 modules: \texttt{xbow-native-lib-etherstub}, \texttt{xbow-native-lib-flow}, \texttt{xbow-native-lib-link}.

			\begin{itemize}
				\item \texttt{create\_etherstub}
				\item \texttt{get\_properties}
				\item \texttt{plumb}
			\end{itemize}

		\end{frame}
	

		\begin{frame}{MBeans}

			Two kinds of objects:
			
			\begin{itemize}

				\item managers (EterstubManager, FlowManager, (V)NicManager)

					entity discovery, creation, deletion

				\item entities (flows, etherstubs, V(NIC)s)

					\begin{itemize}
						\item per-instance attributes management and monitoring
						\item hierarchy reflected in naming (e.g. MBean for \texttt{flow0}
						      created over \texttt{e1000g0} link is registered as \texttt{<domain>:type=Flow,link=e1000g0,name=flow0})
					\end{itemize}
			
			\end{itemize}

			Most often 1:1 MBean method : Native function mapping with return code to exception translation.

		\end{frame}


		\begin{frame}{Graphic User Interface}

				Possible frameworks:

				\begin{itemize}
					\item Eclipse RCP
					\item Jopr
					\item Web frameworks (Spring 3, JSP)
				\end{itemize}

		\end{frame}


\section{Technical issues}

	\begin{frame}{Development and code quality}

		\begin{itemize}
			\item Java and native builds with maven
			\item Unit tests \& mocks for both MBean and native code
			\item Code coverage reports
		\end{itemize}
	
	\end{frame}


\section{Roadmap}

	\begin{frame}{Roadmap}
		
		\begin{itemize}

			\item DONE
			
				C wrappers, MBeans layer $\rightarrow$ management and monitoring possible with JConsole

			\item ONGOING
			
				JIMS integration, collecting statistics

			\item FUTURE
			
				Graphic user Interface, QoS-aware zone migration

		\end{itemize}

	\end{frame}



\end{document}


% vim: enc=utf8 :
