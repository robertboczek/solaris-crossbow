% Plan:
%
% Spis treści
% Spis rysunków
% Słowniczek skrótów
%
% Wstęp (1-2 strony): geneza, obszar, zawartość, dokonania autorów, opis struktury pracy
%
% Rozdziały merytoryczne (4-5, zrównoważone objętościowo):
%   Preambuła (ok. 0.5 strony)
%   Punkty merytoryczne (4-6)
%   Podsumowanie rozdziału
%
%   * opis technologii i uzasadnienie wyboru do rozwiązania problemu
%   * analiza wymagań + projekt systemu (architektura)
%   * opis implementacji, sposób uruchomienia
%   * badania eksperymentalne (tutaj także: profiling, opóźnienia)
%
% Podsumowanie pracy / Zakończenie
%   Wnioski końcowe
%   Co się udało / nie udało (+dlaczego!)
%   Możliwości rozwoju
%
% Spis literatury (numerowany, w tekście _muszą_ być odniesienia, ~20 pozycji)
%
%
% INNE:
%   * całość - ok. 60-70 stron
%   * numeracja - nie więcej, niż 3-poziomowa


\documentclass[11pt]{book}
\usepackage[top=3cm, bottom=4cm]{geometry}
\usepackage[usenames,dvipsnames]{color}


% \usepackage[polish]{babel}
\usepackage[utf8]{inputenc}
\usepackage[T1]{fontenc}
\usepackage{fullpage}
\usepackage[pdfborder={0 0 0}]{hyperref}
\usepackage{float}
\usepackage{graphicx}
\usepackage{scrtime}
\usepackage{tabularx}
\usepackage{listings} 
\usepackage{caption}
\usepackage{color}
% \usepackage[toc,acronym]{glossaries}

\newcommand{\code}[1]{\begin{tt}{#1}\end{tt}}

\newcommand{\reqlabel}[1]{{\textcolor{Red}{\textbf{#1}}\label{#1}}}
\newcommand{\reqref}[1]{\hyperref[#1]{{\textcolor{Red}{\textbf{#1}}}}}

\newcommand{\tasklabel}[1]{{\textcolor{Blue}{\textbf{#1}}\label{#1}}}
\newcommand{\taskref}[1]{\hyperref[#1]{{\textcolor{Blue}{\textbf{#1}}}}}


\title{Component-based system for management of multilevel virtualization of networking resources}
\author{Robert Boczek \and Dawid Ciepliński}

\begin{document}

  \maketitle
    
  \tableofcontents
	

  \chapter{Introduction}


  \chapter{Infrastructure as a Service}

    \section{Benefits}

    \section{IaaS applications}

    \section{Requirements}


  \chapter{Virtualization}

    \section{Operating system-level virtualization}


    \section{Virtualization of networking resources}


  \chapter{Virtual infrastructure construction and deployment}

    \section{Possible approaches}

      \section{Heterogeneous environment}


      \section{Homogeneous environment}

        \subsection{Single node}


        \subsection{Multi-node}


    \section{Problems}

      \subsection{Quality of Service}


      \subsection{Load balancing / Deployment}


      \subsection{Isolation}


      \subsection{Broadcast domain preservation}


  \chapter{Existing virtualization solutions}


  \chapter{Solaris OS as a resource virtualization environment}

    \section{General information}


    \section{Solaris Containers}


    \section{Crossbow}


  \chapter{Component-based software design}


  \chapter{The system overview}

    \section{The JMX-based Infrastructure Monitoring System}


    \subsection{Enabling rich functionality thanks to JIMS integration}


    \section{Architecture}

      \subsection{Main components and their responsibility}


      \subsection{Low-level function access with layer-based design}

        \subsubsection{Benefits}


  \chapter{Case Study}


  \chapter{Implementation details}

    % TODO czy to sie znajdzie?


  \begin{thebibliography}{some-label}
    
    \bibitem[1]{} Crossbow: From Hardware Virtualized NICS To Virtualized Networks, http://conferences.sigcomm.org/sigcomm/2009/workshops/visa/papers/p53.pdf
    \bibitem[2]{} Virtual switching in Solaris, http://hub.opensolaris.org/bin/download/Project+crossbow/Docs/virtualswitch.pdf
    \bibitem[3]{} Oracle Solaris 11 Express Network Virtualization and Network Resource Management, http://www.oracle.com/technetwork/articles/servers-storage-admin/sol11ecrossbow-186794.pdf

  \end{thebibliography}


\end{document}

% vim: et : spelllang=en_us,pl : spell :
