% Plan:
%
% Spis treści
% Spis rysunków
% Słowniczek skrótów
%
% Wstęp (1-2 strony): geneza, obszar, zawartość, dokonania autorów, opis struktury pracy
%
% Rozdziały merytoryczne (4-5, zrównoważone objętościowo):
%   Preambuła (ok. 0.5 strony)
%   Punkty merytoryczne (4-6)
%   Podsumowanie rozdziału
%
%   * opis technologii i uzasadnienie wyboru do rozwiązania problemu
%   * analiza wymagań + projekt systemu (architektura)
%   * opis implementacji, sposób uruchomienia
%   * badania eksperymentalne (tutaj także: profiling, opóźnienia)
%
% Podsumowanie pracy / Zakończenie
%   Wnioski końcowe
%   Co się udało / nie udało (+dlaczego!)
%   Możliwości rozwoju
%
% Spis literatury (numerowany, w tekście _muszą_ być odniesienia, ~20 pozycji)
%
%
% INNE:
%   * całość - ok. 60-70 stron
%   * numeracja - nie więcej, niż 3-poziomowa


\documentclass[11pt]{book}
\usepackage[top=3cm, bottom=4cm]{geometry}
\usepackage[usenames,dvipsnames]{color}


% \usepackage[polish]{babel}
\usepackage[utf8]{inputenc}
\usepackage[T1]{fontenc}
\usepackage{fullpage}
\usepackage[pdfborder={0 0 0}]{hyperref}
\usepackage{float}
\usepackage{graphicx}
\usepackage{scrtime}
\usepackage{tabularx}
\usepackage{listings} 
\usepackage{caption}
\usepackage{color}
% \usepackage[toc,acronym]{glossaries}

\newcommand{\code}[1]{\begin{tt}{#1}\end{tt}}

\newcommand{\reqlabel}[1]{{\textcolor{Red}{\textbf{#1}}\label{#1}}}
\newcommand{\reqref}[1]{\hyperref[#1]{{\textcolor{Red}{\textbf{#1}}}}}

\newcommand{\tasklabel}[1]{{\textcolor{Blue}{\textbf{#1}}\label{#1}}}
\newcommand{\taskref}[1]{\hyperref[#1]{{\textcolor{Blue}{\textbf{#1}}}}}


\title{Component-based system for management of multilevel virtualization of networking resources}
\author{Robert Boczek \and Dawid Ciepliński}

\begin{document}

  \maketitle
    
  \tableofcontents
	

  \chapter{Introduction}

    % QoS, rezerwacja zasobów, izolacja we współczesnych systemach informatycznych


  \chapter{Context}  % TODO ładniej

    \section{Chapter overview}


    \section{QoS-aware networking}


    \section{Resource virtualization approaches}


    \section{Multilevel network virtualization}

      \subsection{Virtual network resources}

      \subsection{Fine-grained QoS control}

      \subsection{Virtual appliances}

      \subsection{,,Network in a box'' concept}


    \section{Applications and benefits of virtual infrastructures}

      \subsection{Testing and simulations}

      \subsection{Improving server-side infrastructure scalability}

      \subsection{Infrastructure as a service}

      \subsection{The role of resource virtualization in the SOA stack}


    \section{Summary}


  \chapter{Requirements analysis}
    
    \section{Chapter overview}

    \section{Functional requirements}

      \subsection{Instantiation}

      \subsection{Discovery}

      \subsection{Accounting}


    \section{Non-functional requirements}

    \section{Underlying environment characteristics}

    \section{General approach and problems it imposes}

      \subsection{Load balancing / Deployment}

      \subsection{Infrastructure isolation}

      \subsection{Broadcast domain preservation}

      \subsection{Constraints}


    \section{Summary}


  \chapter{Solaris OS as a resource virtualization environment}

    \section{Chapter overview}

    \section{General information}

    \section{Lightweight OS-level virtualization with Solaris Containers}

    \section{Crossbow - network virtualization technology}

    \section{Resource access control}

    \section{Summary}


  \chapter{The system architecture}

    \section{Chapter overview}

      % tutaj tez kilka slow i JIMSie i integracji + co to daje


    \section{High-level design}


    \section{System components and their responsibilities}

      \subsection{Assigner}

      \subsection{Supervisor}

      \subsection{Worker}


    \section{Crossbow resources instrumentation}


    \section{Domain model and data flows}


    \section{Summary}


  \chapter{Implementation}
    
    \section{Chapter overview}


    \section{Implementation environment}


    \section{Domain model transformation details}


    \section{Low-level functions access}


    \section{Building and running the platform}


    \section{Summary}


  \chapter{Case Study}

    \section{Chapter overview}

    \section{Clustered GlassFish}

      \subsection{Scenario description}

      \subsection{GlassFish cluster integration}


    \section{Multimedia server}

      \subsection{Scenario description}

      \subsection{Resource access requirements}

      \subsection{Providing tunable and scalable virtual infrastructure}


    \section{Summary}


  \chapter{Summary}

    \section{Chapter overview}

    \section{Conclusions}

    \section{Achieved goals}

    \section{Further work}


  \begin{thebibliography}{some-label}

    % TODO bibtex
    
    \bibitem[1]{} Crossbow: From Hardware Virtualized NICS To Virtualized Networks, http://conferences.sigcomm.org/sigcomm/2009/workshops/visa/papers/p53.pdf
    \bibitem[2]{} Virtual switching in Solaris, http://hub.opensolaris.org/bin/download/Project+crossbow/Docs/virtualswitch.pdf
    \bibitem[3]{} Oracle Solaris 11 Express Network Virtualization and Network Resource Management, http://www.oracle.com/technetwork/articles/servers-storage-admin/sol11ecrossbow-186794.pdf

  \end{thebibliography}


\end{document}

% vim: et : spelllang=en_us,pl : spell :
